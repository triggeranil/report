\documentclass{article}
\usepackage{graphicx}
\begin{document}
\title{Probability Hardware Assignment}
\author{Anil Nayak || AI22BTECH11019}
\maketitle

\section{Introduction}
The objective of this experiment is to create a circuit that displays random numbers on a screen when a USB connector is connected. The circuit utilizes various components such as capacitors, resistors, wires, ICs, a screen, and a USB connector. The functionality of the circuit is achieved through a microcontroller or programmable logic device that generates the random numbers and controls the display module.
\section{Abstract}
This report describes the design and implementation of a circuit that displays random numbers when a USB connector is connected. The circuit utilizes a breadboard, capacitors, resistors, wires, five specific ICs, a screen, and a USB connector. The circuit incorporates a microcontroller to generate the random numbers and send display commands to the screen.

\section{Materials and Components}
The following materials and components are required for the experiment:
\begin{itemize}
  \item Breadboard
   \item Five specific ICs (please specify the IC names and functions)
  \item Screen module (compatible with ICs or a separate microcontroller)
  \item Capacitors (specific values as per IC datasheets)
  \item Resistors (specific values as per IC datasheets)
  \item Wires for connecting the components
  \item USB connector
\end{itemize}

\section{Circuit Design}
The circuit design involves the following steps:
\begin{enumerate}
  \item Identify the pin configurations and functions of the five specific ICs.
  \item Refer to the datasheets of the ICs to determine the required capacitors and resistors.
  \item Place the ICs on the breadboard and connect the power and ground pins to appropriate power sources.
  \item Connect the capacitors and resistors as per the recommended configurations in the datasheets.
  \item Establish the connections between the ICs, microcontroller, and screen module using wires.
  \item Connect the USB connector to the appropriate pins of the microcontroller.
\end{enumerate}

\section{Programming the Microcontroller}
\begin{enumerate}
  \item Select a suitable microcontroller or programmable logic device that can generate random numbers.
  \item Write the necessary code to generate random numbers within the desired range.
  \item Incorporate the code to control the display module and send the random numbers for display.
  \item Compile and upload the code to the microcontroller.
\end{enumerate}

\section{Testing and Results}
\begin{enumerate}
  \item Connect the USB cable to the USB connector of the circuit.
  \item Power on the circuit and observe the screen for the display of random numbers.
  \item Verify if the numbers displayed are truly random and within the expected range.
  \item Repeat the testing process multiple times to ensure the consistency and reliability of the circuit.
\end{enumerate}

\section{Conclusion}
In this experiment, a circuit was successfully designed and implemented to display random numbers when a USB connector is connected. The circuit utilized capacitors, resistors, wires, specific ICs, a screen module, and a USB connector. The microcontroller was programmed to generate random numbers and control the display module. Through testing, it was confirmed that the circuit displayed random numbers as intended.

\section{Future Improvements}
Further enhancements to the circuit can be considered, such as incorporating user input to control the range of random numbers or adding additional features like a menu system or animations on the display. Additionally, the circuit can be miniaturized and integrated into a custom PCB for compactness and ease of use.


\begin{figure}[htbp]
  \centering
  \includegraphics[width=0.5\textwidth]{image}
  \caption{output}
  \label{fig:image_label}
\end{figure}


\end{document}

